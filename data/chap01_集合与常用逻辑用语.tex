\chapter{集合与常用逻辑用语}
\section{集合的概念}
\marginnote{2024-7-19}
\begin{intro}[教学片段\cite{HouCongShuXueHaoWanDaoWanHaoShuXueGaoZhongShuXueDiYiKeBuFangCongWan2020}]
面对新的教师,新的同学,学生往往比较拘谨,不敢表达,这无疑会影响教师教育教学的质量.故而,教师在课程导入时,不妨以集合的“确定性”为背景和学生开一个小玩笑.班长:起立!学生:老师好!教师笑语:同学们好!请所有长头发的同学站一下,其余同学请坐下.(停顿几秒给学生思考)教师继续:(目测包含所有站着的学生)请站着的同学中身高低于2米的坐下.
\end{intro}
\begin{purpose}
意在建立学生对集合概念的感性认识.“长头发”的不确定性引发学生思考:多长的头发算长头发?部分头发较长的女生不知所措,有的男生也在纠结自己的坐与站的问题.教师快速追加坐下的条件“身高低于2米”,则所有站立的学生坐下,化解了这一部分学生的尴尬.课堂上,师生谈笑间拉近了心理的距离.
\end{purpose}

\begin{intro}
    教师开始本节课的导入语:首先祝贺大家升入高中的学习,老师刚刚以高中数学中“集合”的概念为背景和大家开了一个小玩笑.什么是集合?我们全班的全体同学可以是一个集合,全体的男同学或全体的女同学也都是一个集合,但所有长头发的同学不能构成一个集合,因为集合要满足其内在的个体(元素)是确定的.这也是引发部分同学不确定自己是坐还是站的原因,而站着的同学中“身高低于2米的同学”显然是能够确定的,它是一个集合.集合是现代数学的一个基本概念,也是我们高中数学学习的第一个概念.同学们在后续的学习中要注意数学概念的学习,概念的理解程度决定了我们数学学习的深度,也直接决定了数学解题水平的高低.
\end{intro}
\begin{purpose}
通过学生不难理解的集合概念,让学生体会数学概念在数学学习中的重要性.教学中,教师不必对集合的概念加以定义,以学生的感性认识理解即可,意在提升学生的学习兴趣和数学认知,也可留下悬念为后续正式学习集合相关知识铺垫情感因素.
\end{purpose}

\section{集合的表示方法}