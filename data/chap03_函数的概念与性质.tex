\chapter{函数的概念与性质}
\section{函数的概念}
\newpage
\section{函数的奇偶性}
\subsection{教材分析}
本节课\cite{YuAPOSSiDuanShiJiaoXueSheJiKeTiHanShuDeQiOuXing2024}所用教材为2019年人教A版《普通高中教科书 $\cdot$ 数学》, 选择的教案内容为其中的第 3 章(《函数概念与性质》) 的3.2.2节, 探讨函数奇偶性特征.

本章在开始之初, 首先简单地介绍了函数的基本概念(即函数反映变量关系而构建起的数学模型) , 然后在此基础上探讨了函数的基本性质(单调性) . 在本节中, 重点是探讨函数的奇偶性, 为后续研究幂函数、对数函数等多种函数性质奠定基础, 也能方便后续对相关函数性质的深刻理解.

\subsection{教学目标}
\begin{enumerate}
	\item 指导学生能够通过数量关系来判断出函数图像的对称类型: 是原点对称,  $x$ 轴对称还是 $y$ 轴对称, 然后在此基础上明确奇偶性的相关定义.
	\item 学生能够根据自己掌握的知识, 对函数的奇偶性与否作出判断.
	\item 要求学生能够把握数形结合思想精髓, 能够遵循由特殊到一般(具体到抽象) 的思考和研究过程, 提升个体的思维能力, 深刻感受函数性质的相关研究适用方法.
\end{enumerate}

\subsection{学情分析}
本节课是针对高一学生开展授课, 学生具备的知识、方法和能力特征分别如下:

(1) 具备深入探究的知识基础;

(2) 具备数量关系刻画方法;

(3) 能够从数量关系层面简单地刻画函数对称性, 但对数形转化过程不够清晰. 最终确定了该节课的主要教学内容, 明确了其中的难点和重点内容, 即了解 $f(-x)=$ $f(x)$ (或 $f(-x)=-f(x)$ )的表达意义.

\subsection{教学方法}
本教学设计以 APOS 理论为基础, 根据该理论的 4 个阶段来完成相关教学设计过程.
\begin{enumerate}
	\item 活动阶段: 设置学生自主探究活动, 让学生在实际探究过程中获得对所学知识点及其背景的直观了解, 方便本文后续的函数奇偶性教学的顺利开展.
	\item 过程阶段: 深入思考活动阶段的探究内容, 鼓励学生在头脑中对其进行详细的描述, 将其转化为思维的内化内容, 并对活动中的相关内容进行总结反思, 引导学生能够通过抽象的方法揭示出函数奇偶性的基本特征.
	\item 对象阶段: 经历上述两个阶段以后, 学生对基础性概念的理解日益完善, 能够通过形式化的符号和定义来抽象出函数奇偶性的本质特征.
	\item 图式阶段: 在这一阶段中, 需要引导学生能够在旧的知识基础上, 与新的知识建立起基本联系, 能够将函数奇偶性的相关概念、特征与其他的定理和性质构建起内在脉络, 形成有关结构性的知识导图.
\end{enumerate}

\subsection{教学过程}
\subsubsection{活动阶段}
% 活动1
\begin{activity}
	让学生观察大自然和生活中存在的轴对称与中心对称图形, 然后鼓励学生根据自己的知识经验来列举出在学习或生活中碰见的相关图形, 能够将学生的观察和想象与实际的教学情景衔接, 从而顺利地实现学生的自主探究向相关教学内容的顺利过渡.

	问题 1 : 说出展示的图形分别属于哪种对称图像?

	问题 2 : 你在生活中还能遇到哪些中心或轴对称图形?

	% \includegraphics[max width=\textwidth, center]{2024_10_21_5f8330c8eaf7c7eac80fg-2}

	问题 3 : 数学中哪些函数的图像具有对称性. 问题 4 : 举出具有对称性的函数的解析式.
\end{activity}
% 设计目的
\begin{purpose}
	设计生动有趣的数学问题, 帮助学生在现有的认知基础上, 利用一些推理和判断方法, 寻找到对称性图像的本质特征, 激发学生的学习兴趣和内在动力.
\end{purpose}

% 活动2
\begin{activity}
	引导学生用"形"探究.

	问题 5 : 指出 $y=x^{2}, y=|x|$ 图像具备哪些共同特征?

	问题6:如何证明函数图像是 $y$ 轴对称图像?

	问题7: 图像是由点构成的, 图像关于 $y$ 轴对称, 那么在 $y$ 轴两侧的图像中相应点坐标之间存在什么关系?

	问题8: 通过直尺测量的方式, 告诉我们这两个点在横纵坐标上关系究竟如何?

	问题9: 是否还能够发现其他对称点的对称关系?
\end{activity}
% 设计目的
\begin{purpose}
	让学生对函数图像产生直观感知, 引发学生的独立思考和主动探索, 通过测量或折纸的方式调动学生参与具体教学的积极性, 形成良好的自主探索兴趣, 从而揭示出偶函数性质特征.
\end{purpose}

% 活动3
\begin{activity}
	引导学生用"数"探究.

	问题 10 : 根据计算得到的数值填写完成下表 1 , 并观察数值两者之间的关系.

	表 1 : 函数 $f(x)=x^{2}$

	\begin{center}
		\begin{tabular}{|c|c|c|c|c|c|c|c|c|c|}
			\hline
			$x$ & $\ldots$ & -3 & -2 & -1 & 0 & 1 & 2 & 3 & $\ldots$ \\
			\hline
			$y$ & $\ldots$ & 9  & 4  & 1  & 0 & 1 & 4 & 9 & $\ldots$ \\
			\hline
		\end{tabular}
	\end{center}

	表 2 : 函数 $f(x)=\mathbf{2}-|x|$

	\begin{center}
		\begin{tabular}{|c|c|c|c|c|c|c|c|c|c|}
			\hline
			$x$ & $\ldots$ & -3 & -2 & -1 & 0 & 1 & 2 & 3  & $\ldots$ \\
			\hline
			$y$ & $\ldots$ & -1 & 0  & 1  & 2 & 1 & 0 & -1 & $\ldots$ \\
			\hline
		\end{tabular}
	\end{center}

	问题 11: 关于 $y$ 轴对称的每一组点, 是否都有 $x$ 互为相反数,  $y$ 相等的特征?

	问题 12 : 这两个函数的定义域是什么? 是否存在对称关系?

	问题 13 : 对表 1 和表 2 进行观察, 告诉我们函数图像特征.

	问题 14: 对于任意的 $a \in \mathbf{R}$ ( $a$ 为常数), $f(-a)=f(a)$ 是否存在?

	问题 15: 对于任意 $x$ (定义域内) ,  $f(-x)=f(x)$ 是否存在?

	问题 16: 任意一个关于 $y$ 轴对称的函数图象,  $f(-x)=$ $f(x)$ 是否成立?

	问题 17: 对于定义域内任意一个 $x, f(-x)=f(x)$ 恒成立, 是不是就意味着该函数为 $y$ 轴对称函数?

\end{activity}
% 设计目的
\begin{purpose}
	此次教学设计的主要意图在于帮助学生认识到数学教学不能简单地理解为数学知识教学(即单纯地传播数学知识点) , 而是一种帮助学生养成数学思维, 促进学生发展, 捕捉数学价值的教育方式.

	数学教学的核心在于数学思想的灌输, 包括函数与方程、数形结合、分类讨论、化归与转化、特殊到一般等. 上述设计的 5 个问题, 能够帮助学生逐渐从动眼动手向动脑动心转化, 让学生在探索的过程中总结数学规律, 揣摩数学思想, 掌握由具体到抽象的思维过程.
\end{purpose}


\subsection{过程阶段}
"过程阶段"是对"活动阶段"进行的外显数学活动的进一步思考. 学生通过对上一阶段活动的重复参与, 能够在脑海中形成一系列复杂的数学思想过程, 进行相对深层次的抽象思维归纳和推理分析, 从而把握教学内容的本质特征.

\subsubsection{偶函数定义的引入}
以 $f(x)=x^{2}, f(x)=2-|x|$ 函数为例, 在进行相关定义时, 首先要明确函数定义域, 然后在此基础上来探索两个函数关于原点对称的特征, 同时又发现 $f(-x)=f(x)$ , 于是将其称作为偶函数.

讨论 1 : 结合前文的推导过程开展广泛的小组讨论, 鼓励学生自行归纳出偶函数的定义.
\begin{purpose}
	学生通过前面的思维和探索步骤自主完成了偶函数的定义过程, 是本章节教学内容的核心部分, 同样也是检验前期探索过程的重要途径.
\end{purpose}

\subsubsection{类比联想, 得出奇函数定义}
经过了对偶函数的探究和定义过程, 就是应该放手让学生自主探究奇函数的定义, 鼓励学生通过类比的方法得到相关的准确定义描述.
\begin{purpose}
	教师要充分地信任学生, 敢于放手交给学生自主探索, 而不是想尽方法来限制学生, 影响学生的思维能力的发挥. 学生在进行奇函数定义探索时, 可以完全照搬偶函数的整个探索过程, 然后自主得出相关的函数定义.
\end{purpose}

\subsection{对象阶段}
在这一阶段, 需要将学生探究出的奇偶函数本质特性通过形式化的符号和定义表达出来, 使之成为具体对象, 方便后续的深入研究.
\subsubsection{板书偶函数的定义}
\begin{tcolorbox}[colback=white, colframe=black, rounded corners, boxrule=0.1mm]
	一般地, 设函数 $f(x)$ 的定义域为 $D$ , 如果 $\forall x \in D$ , 都有 $-x \in D$ , 且 $f(-x)=f(x)$ , 那么 $f(x)$ 就叫做偶函数.
\end{tcolorbox}

追问 1: " $\forall x \in D$ , 都有 $-x \in D "$ 是什么意思?

追问 2: $" \forall x \in D$ , 都有 $-x \in D "$ 可以省略吗?

\begin{purpose}
	学生对奇偶性的理解容易浮于表面, 所以教师应该引导学生关注关键词, 从而深刻、透彻地理解概念. 通过数学符号化能够使得数学教学更加清晰简洁和准确, 也更方便学生学习和记忆理解相关概念.
\end{purpose}

\subsubsection{板书奇函数的定义}
通过对偶函数定义进行充分理解以后, 对奇函数定义进行板书, 引导学生在脑海中勾画和理解相关函数概念. 完成相关探索工作以后, 教师可以简要地介绍一些背景知识、相关的数学家(欧拉等) , 同时还能够揭示出它与幂函数之间的历史渊源.

\begin{purpose}
	注重将数学史与数学教学相融合, 让学生感受到数学知识学习过程中的乐趣, 掌握函数奇偶性的基本概念构成(内涵、外延和现实价值) , 能够积极主动地参与现实世界中数学规律的发现和探索过程.
\end{purpose}

\subsection{图式阶段}
这个阶段的重点和核心在于与其他相关概念之间建立紧密的联系, 从而形成相对完善的综合图式, 并成功地将其划归到自己的认知版图当中, 建立起新旧知识的内在联系.
\subsubsection{函数奇偶性定义对比}
\begin{center}
	\begin{tabularx}{\textwidth}{|c|X|X|}
		\hline
		特征  & 偶函数                                 & 奇函数                      \\
		\hline
		相同点 & 定义域关于原点对称,基于定义域而做出的描述.               & 定义域关于原点对称,基于定义域而做出的描述.    \\
		\hline
		不同点 & y 轴对称的图像特征; 当自变量取一对相反数时,相应的两个函数值相等.  & 图像关于原点对称; 自变量相反函数值也为相反数.  \\
		\hline
	\end{tabularx}
\end{center}

\begin{purpose}
	让学生比较奇函数与偶函数的异同, 在比较中反映学生对定义的理解, 对所学的内容进行整理、归类、改造和创造, 促进知识间的整合.
\end{purpose}

\subsubsection{应用概念, 自我尝试}
\begin{exercise}
	对下述函数奇偶性进行判定.

	(1) $f(x)=x^{4}-x^{2}+1$ ;(2) $f(x)=x-\frac{1}{x}$ ; \\
	(3)  $f(x)=\frac{x^{3}-x^{2}}{x-1}$ ; (4) $f(x)=\sqrt{1-x^{2}}+\sqrt{x^{2}-1}$ .
\end{exercise}
\begin{purpose}
	为了明确函数奇偶性的4大分类, 除了奇偶函数两类以外, 还包括非奇非偶和既奇又偶函数两种类型.
\end{purpose}

\subsubsection{变式练习, 强化训练}
\begin{exercise}
	若函数 $f(x)=a x^{2}+2 x^{3}-x, x \in(2 b-1, b)$ 是奇函数, 求 $a+2b$ .
\end{exercise}
\begin{purpose}
	通过对函数的奇偶性的准确判定以后, 结合定义域, 通过适当的变式训练, 能够帮助学生掌握一些初等函数的奇偶特征, 并且能够牢记其中的奇偶函数分类, 便于后续的直接应用.
\end{purpose}

\subsubsection{课堂总结, 深化理解}

总结 1 : 奇偶函数在性质特征上存在着哪些相同或不同点.

总结 2 : 通过定义如何对函数的奇偶性进行判定, 应当遵循几个步骤.

总结 3 : 通过在某一定义域对奇偶函数的增减函数性质进行定义, 明确在对称定义区间的单调性特征.

总结 4 : 经过此次课程的学习, 学生对相关数学思想有何看法, 对数学方法有何种理解.

\begin{purpose}
	通过设计相关的总结题项, 让学生能够通过比较的方式来把握奇偶函数之间的异同点, 同时还能够掌握基本的判定技巧和方法步骤, 然后在此基础上将函数单调性特征引入其中, 构建两者之间的内在联系, 进一步强化学生对相关概念的深刻理解和准确认知.
\end{purpose}

\subsubsection{分类作业, 巩固提高}
作业1: 练习册 p$47-48$ .

作业2: 课后搜集查找有关函数奇偶性的资料, 结合所学的概念和习题, 完成一份学习报告, 一周后上交.

作业 3 : 对于函数 $f(x)=x^{4}+2|x|$ , 能否通过添加项, 使其依然为偶函数, 或是变为奇偶函数?

\begin{purpose}
	在设计相关作业习题时, 要充分考虑到学生的学习成效. 针对不同的学生设定不同的作业内容, 即要做到分层作业布置, 这样才能够确保处于不同层次的学生都能够在自己的知识体系中有所发展, 同时作业的内容还不能脱离教材中的基础知识并进行适当的思维难度提升.
\end{purpose}